\documentclass[13pt,]{tufte-handout}

% ams
\usepackage{amssymb,amsmath}

\usepackage{ifxetex,ifluatex}
\usepackage{fixltx2e} % provides \textsubscript
\ifnum 0\ifxetex 1\fi\ifluatex 1\fi=0 % if pdftex
  \usepackage[T1]{fontenc}
  \usepackage[utf8]{inputenc}
\else % if luatex or xelatex
  \makeatletter
  \@ifpackageloaded{fontspec}{}{\usepackage{fontspec}}
  \makeatother
  \defaultfontfeatures{Ligatures=TeX,Scale=MatchLowercase}
  \makeatletter
  \@ifpackageloaded{soul}{
     \renewcommand\allcapsspacing[1]{{\addfontfeature{LetterSpace=15}#1}}
     \renewcommand\smallcapsspacing[1]{{\addfontfeature{LetterSpace=10}#1}}
   }{}
  \makeatother

\fi

% graphix
\usepackage{graphicx}
\setkeys{Gin}{width=\linewidth,totalheight=\textheight,keepaspectratio}

% booktabs
\usepackage{booktabs}

% url
\usepackage{url}

% hyperref
\usepackage{hyperref}

% units.
\usepackage{units}


\setcounter{secnumdepth}{-1}

% citations

% pandoc syntax highlighting

% longtable
\usepackage{longtable,booktabs}

% multiplecol
\usepackage{multicol}

% strikeout
\usepackage[normalem]{ulem}

% morefloats
\usepackage{morefloats}


% tightlist macro required by pandoc >= 1.14
\providecommand{\tightlist}{%
  \setlength{\itemsep}{0pt}\setlength{\parskip}{0pt}}

% title / author / date
\title{How Do Self-Employed Software Developers Invest in Their Career?}
\author{Philip Morgan}
\date{November 5, 2019 - v1.0}


\begin{document}

\maketitle

\begin{abstract}
\noindent A study of how self-employed software developers invest in their
careers, and where they believe career opportunity comes from, by Philip
Morgan Consulting.

The question that motivated this study: How do self-employed software
developers invest in their career? The answer: It's pretty much what you
might think, but there are some surprises.
\end{abstract}


{
\hypersetup{linkcolor=black}
\setcounter{tocdepth}{2}
\tableofcontents
}

\newpage

\hypertarget{key-findings}{%
\section{Key Findings}\label{key-findings}}

\begin{enumerate}
\def\labelenumi{\arabic{enumi}.}
\tightlist
\item
  My study participants -- self-employed software developers -- strongly
  prefer individual, asynchronous learning styles (books and courses,
  primarily) over group, synchronous styles (realtime courses or
  mentoring). IRL conferences are the primary exception to this pattern.
\item
  New business opportunity for this group comes from relationships
  (networking and past clients), paying attention to changes in the
  market, and persistence in pursuing opportunity.
\item
  Two currently ``hot'' career investment tools (listening to podcasts
  and mentorship) are used at surprisingly low levels within my sample
  group.
\item
  While speaking and publishing are reliable opportunity-generators,
  they are used at relatively low levels by my sample group.
\end{enumerate}

\hypertarget{three-questions}{%
\section{Three questions}\label{three-questions}}

There are three specific questions from the survey that I'll focus on in
this report

\hypertarget{how-do-you-invest-in-technical-skills}{%
\subsection{1) How do you invest in technical
skills?}\label{how-do-you-invest-in-technical-skills}}

\marginnote{The full question was: Please list ways you have you spent time and money for developing your technical skills.}

I standardized these responses to survey questions using a simple coding
system: a response starting with ``a-'' is an action, while one starting
with ``s-'' is a sentiment or belief.

What you'll see below are the top 3 responses from each of my sample
groups (one was recruited from LinkedIn, the other from my email list,
more details in the Methodology section), and how many of my respondents
indicated that action, sentiment, or belief, and what percentage of the
total number of survey completions that number represents (rounded to
the nearest whole number).

\begin{longtable}[]{@{}ll@{}}
\toprule
LinkedIn Sample & List Sample\tabularnewline
\midrule
\endhead
a-online-courses - 12 - 55\% & a-online-courses - 22 -
67\%\tabularnewline
a-reading-books - 4 - 18\% & a-reading-books - 16 - 48\%\tabularnewline
a-irl-conferences - 4 - 18\%" & a-coding-practice - 9 -
27\%\tabularnewline
\bottomrule
\end{longtable}

\newpage

\hypertarget{how-do-you-invest-in-business-skills}{%
\subsection{2) How do you invest in business
skills?}\label{how-do-you-invest-in-business-skills}}

\marginnote{The full question was: Please list ways you have you spent time and money for business or self-employment skills.}

\begin{longtable}[]{@{}ll@{}}
\toprule
LinkedIn Sample & List Sample\tabularnewline
\midrule
\endhead
a-reading-books - 2 - 9\% & a-online-courses - 15 - 45\%\tabularnewline
a-learning - 2 - 9\% & a-reading-books - 15 - 45\%\tabularnewline
a-experimenting - 2 - 9\% & a-email-lists - 6 - 18\%\tabularnewline
\bottomrule
\end{longtable}

\hypertarget{where-does-opportunity-come-from}{%
\subsection{3) Where does opportunity come
from?}\label{where-does-opportunity-come-from}}

\marginnote{The full question was: Consider your entire career as a self-employed software developer and times you have gotten new opportunities, better projects, or other forms of career improvement. What do you think led to these improvements in your career?}

\begin{longtable}[]{@{}ll@{}}
\toprule
LinkedIn Sample & List Sample\tabularnewline
\midrule
\endhead
a-networking - 7 - 32\% & a-networking - 8 - 24\%\tabularnewline
s-perceptiveness - 3 - 14\% & a-relationships - 5 - 15\%\tabularnewline
s-persistence - 2 - 9\% & a-relationship-with-existing-clients - 4 -
12\%\tabularnewline
\bottomrule
\end{longtable}

\hypertarget{surprises}{%
\section{Surprises}\label{surprises}}

\begin{quote}
Q: How do you invest in your career?

A: I buy books, courses, attend events, practice what I learn from those
sources, and invest in my network to make sure I have future
opportunity.
\end{quote}

That answer is about the least surprising possible answer to the ``how
do you invest'' question. And it is a good summary of what my survey
responses indicate.

As I said at the top of this report, ``it's pretty much what you'd
think''. There are indeed few surprises in the top 3 responses to each
question.

I'm also not surprised at the methods of career investment that were on
the list but were relatively unpopular, specifically:

\begin{itemize}
\tightlist
\item
  Publishing
\item
  Speaking
\end{itemize}

These are well-established -- we might use the over-used word
\emph{proven} -- methods of gaining visibility and cultivating
authority. But they're also not easy to use. It's easier to read a book
than write one -- or even a well-executed blog post -- and it's easier
to consume a talk than get yourself on the stage and give a talk.

Despite the effectiveness of these career investment tools, it's no
surprise that they're close to the bottom of the pile in terms of
popularity. They're a ton of work. In fact, their relative unpopularity
might explain their continued effectiveness: in a world where nearly
everybody can give a killer talk, killer talks are not going to help you
stand out. But in the actual world we live in, killer talks are rare,
and lead to visibility and authority.

Here's another interesting way to interrogate this data: \emph{what's
not on the list of responses at all, or what's much further towards the
bottom than we'd expect?}

Twitter gives us a certain filtered picture of the world of software
development, and my Twitter feed currently has a lot of content about
the value of mentorship, along with heated blowback against companies
that place low value on mentorship. Yet, in my research, mentorship is
represented either very minimally (1 of 33 list respondents) or not at
all (my LinkedIn sample) as a career investment method.

This is a surprising disconnect between the picture of the world Twitter
offers, and the picture my research offers, which portrays software
developers as very motivated to use individual learning methods (books,
courses) and not very motivated at all to use more collaborative
learning methods (mentorship, pair programming, online groups).

To be fair, the Twitter hype train around mentorship is clearly in the
context of regular employment, not self-employment. Yet if mentorship
was broadly valuable, wouldn't self-employed people -- who enjoy
relatively more freedom in the design of their work and career and
therefore might have more ability to forge mentorship relationships --
seek it out more than is represented in my data? Is mentorship's value
highest in the world of regular employment and relatively lower in
self-employment?

Listening to podcasts -- as a career investment tool -- feels
underrepresented in my data. I say ``feels'' because I haven't measured
how often my friends, colleagues, clients, and the press mention
listening to podcasts, but going on my subjective experience alone, I'd
give it a top-3 ranking among possible methods of career investment. Yet
my data ranks it lower. Not dramatically lower, mind you. In my list
sample, it ranked 8 and 4 respectively among methods for investing in
technical and career skills, and did not appear at all in my LinkedIn
sample. Even so, the disparity between my subjective experience of
hearing ``podcasting, podcasting, podcasting'' and my somewhat more
objective experience of seeing it represented at a relatively low level
in my data is jarring, and leads to further questions. \footnote{Chief
  among these questions would be whether there's a correlation between
  the activity of podcast listening and some measurable career outcome
  or achievement. Another question would be whether books and courses
  are viewed as ``pure'' learning tools while podcasts might be seen
  more as ``info-tainment'' tools.}

\hypertarget{insights}{%
\section{Insights}\label{insights}}

What insights emerge from this slice of the data? There are at least
two.

\hypertarget{insight-1-opportunity-comes-from-relationships-and-market-awareness.}{%
\subsection{Insight 1: Opportunity comes from relationships and market
awareness.}\label{insight-1-opportunity-comes-from-relationships-and-market-awareness.}}

As I often like to say: unfortunately, you're in a relationship
business. My survey data bears this out.

In fact, when I dig into the full-resolution picture of where my
respondents believe opportunity comes from, a clear pattern emerges:

\textbf{Relationship assets outperform technical skill assets.}

Below is the full ranked list of coded responses to the question:
``Consider your entire career as a self-employed software developer and
times you have gotten new opportunities, better projects, or other forms
of career improvement. What do you think led to these improvements in
your career?''

I've bolded the responses that indicate relationship assets, and
italicized those that indicate technical skill assets. The number to the
right of each line is the number of respondents who indicated that
action or sentiment:

\begin{longtable}[]{@{}ll@{}}
\toprule
LinkedIn Sample & List Sample\tabularnewline
\midrule
\endhead
\textbf{a-networking - 7} & \textbf{a-networking - 8}\tabularnewline
\textbf{s-perceptiveness - 3} & \textbf{a-relationships -
5}\tabularnewline
s-persistence - 2 & \textbf{a-relationship-with-existing-clients -
4}\tabularnewline
\emph{s-focus-on-quality - 2} & s-luck - 4\tabularnewline
\textbf{s-connections - 2} & a-focus-on-expensive-problems -
3\tabularnewline
\emph{s-skill - 1} & s-confidence - 3\tabularnewline
\emph{s-problemsolving-ability - 1} & \emph{a-certification -
2}\tabularnewline
s-positive-attitude - 1 & a-building-confidence - 2\tabularnewline
s-openness - 1 & a-intentional-client-work - 2\tabularnewline
\textbf{s-networking - 1} & a-marketing - 2\tabularnewline
s-longevity - 1 & a-mentoring - 2\tabularnewline
\emph{s-legacy-orientation - 1} & a-sales - 2\tabularnewline
s-focus - 1 & \emph{a-skill-growth - 2}\tabularnewline
s-curiosity - 1 & a-being-where-opportunities-are - 1\tabularnewline
\emph{s-continuous-improvement - 1} & \emph{a-continuous-improvement -
1}\tabularnewline
s-coincidence - 2 & a-discipline - 1\tabularnewline
s-challenging-competetive-environment - 1 & \emph{a-doing-good-work -
1}\tabularnewline
a-speaking - 1 & a-exceeding-comfort-zone - 1\tabularnewline
a-recruiters - 1 & a-experiential-learning - 1\tabularnewline
a-publishing - 1 & \emph{a-focus-on-hot-technologies - 1}\tabularnewline
& a-niche-visibility - 1\tabularnewline
& a-podcasts - 1\tabularnewline
& a-portfolio-or-case-studies - 1\tabularnewline
& a-publishing - 1\tabularnewline
& a-reading-books - 1\tabularnewline
& a-risk-taking - 1\tabularnewline
& a-service - 1\tabularnewline
& a-specializing - 1\tabularnewline
& \emph{a-tech-skill - 1}\tabularnewline
& a-visibility - 1\tabularnewline
& s-divine-intervention - 1\tabularnewline
& s-mindset - 1\tabularnewline
& s-tough-streak - 1\tabularnewline
 &\tabularnewline
\bottomrule
\end{longtable}

In the LinkedIn sample the differentiation between relationship and
technical skill assets is not as obvious as in my email list sample, but
the pattern is there in both samples: relationship assets (networking,
maintaining relationships with past clients, and actively building a
network) are more commonly listed as sources of opportunity than
technical skills.

An interesting and relevant followup study would focus on the ROI of
some of these career assets. This would add nuance that's currently
missing from my data.

\hypertarget{insight-2-we-have-to-wonder-are-devs-choosing-the-most-effective-learning-modes}{%
\subsection{Insight 2: We have to wonder, are devs choosing the most
effective learning
modes?}\label{insight-2-we-have-to-wonder-are-devs-choosing-the-most-effective-learning-modes}}

My data shows a clear preference for individual, asynchronous means of
learning or career investment. For example, on every question focused on
career investment, books, online courses, and individual practice are
the most popular forms of investment. These are all generally done on an
individual basis, and can be consumed asynchronously. In one instance
(the LinkedIn Sample, the question about investing in technical skills),
the non-individual, synchronous mode of IRL conferences crept into the
top 3. Given that every other question elicited a top-3 of individual,
asynchronous modes, this makes IRL conferences -- and group, synchronous
modes in general -- seem like a bit of an outlier.

This raises the question: are individual, asynchronous learning modes
the best ones devs could be using to invest in their careers? An
efficient markets perspective would respond: ``Absolutely. Your
respondents aren't dumb, and they have the best data about what will
produce good results for their careers, and they've incorporated this
data into their decisions about how to invest.''

An innovation mindset would respond: ``Maybe. But maybe there are more
effective modes this group isn't using because these more effective
modes aren't available, understood, or sufficiently adapted for the
needs of software developers.'' For now, this remains an open question,
and one that invites further study.

\hypertarget{method}{%
\section{Method}\label{method}}

In May 2019, I surveyed two groups of self-employed software developers.
One group was a convenience opt-in sample from LinkedIn, and the other
was a convenience opt-in sample from my \textasciitilde{}2,000-member
email list. I screened both samples by asking whether participants had
spent time or money on career development in the last 2 years.

\begin{itemize}
\tightlist
\item
  22 people from LinkedIn completed the survey.
\item
  33 people from my email list completed the survey.
\item
  The median age of respondents is 42 (LinkedIn sample) and 41 (list
  sample).
\item
  64\% of the LinkedIn sample have invested in their career in some way.
  36\% have not.
\item
  95\% of the email list sample have invested in their career in some
  way. 5\% have not.
\end{itemize}

\hypertarget{open-data}{%
\section{Open data}\label{open-data}}

You can inspect and use for your own purposes the anonymized survey
responses generated by this research:
\url{https://docs.google.com/spreadsheets/d/1QqOS6iqMEmk96sEUbNb6hm59PgQsCt9piSm82g19zDE/edit\#gid=0}

\hypertarget{discussion}{%
\section{Discussion}\label{discussion}}

I welcome discussion with you about the conclusions you might draw from
this data, especially if they differ from mine. 

\hypertarget{brought-to-you-by-philip-morgan-consulting}{%
\section{Brought To You By Philip Morgan
Consulting}\label{brought-to-you-by-philip-morgan-consulting}}

\includegraphics{https://pmc-dropshare.s3-us-west-1.amazonaws.com/New-Logo-PMC4.png}

I'm Philip Morgan, and I'm focused on helping dev shops cultivate,
commercialize, and monetize expertise.

There's an underlying tension inherent in this expertise project: what
dev shops need to do to cultivate exceptionally valuable expertise is
often in direct conflict with what they need to do to serve clients --
and generate revenue -- today. My work helps resolve this tension.

Learn more: \url{https://philipmorganconsulting.com/research/}



\end{document}
